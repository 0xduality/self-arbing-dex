\documentclass{article}

% Definitions of handy macros can go here
\usepackage{xcolor}
\definecolor{myblue}{HTML}{0060c0}
\definecolor{myred}{HTML}{c00060}
\definecolor{mygreen}{HTML}{60c000}

\usepackage{hyperref}

\hypersetup{
  linkcolor   = myblue,
  citecolor   = myred,
  urlcolor    = mygreen,
  colorlinks  = true,
  pagebackref = true,
}

\usepackage[utf8]{inputenc} % allow utf-8 input
\usepackage[T1]{fontenc}    % use 8-bit T1 fonts
\usepackage{url}            % simple URL typesetting
\usepackage{booktabs}       % professional-quality tables
\usepackage{nicefrac}       % compact symbols for 1/2, etc.
\usepackage{microtype}      % microtypography

% --------------------------------
% Additional packages and commands
\usepackage{amsmath,amsfonts,amsthm,amssymb}
\usepackage[]{xcolor}
\usepackage[numbers]{natbib}
\usepackage{comment}
\usepackage{graphicx} 
\usepackage{algpseudocode}
\usepackage{algorithm}

\newcommand{\R}{\mathbb{R}}
\newcommand{\E}{\mathbb{E}}
\newcommand{\I}{\mathbb{I}}
\newcommand{\N}{\mathbb{N}}

\DeclareMathOperator*{\argmin}{argmin}
\DeclareMathOperator*{\argmax}{argmax}
\newtheorem{thm}{Theorem}

\makeatletter
\renewcommand\tableofcontents{%
    \@starttoc{toc}%
}
\makeatother
 
\begin{document}

\title{The Self-Arbing DEX}

\author{Anonymous Author}

\maketitle

\begin{abstract}
We describe how a DEX can include backrunning logic 
that can be used to keep substantial arbitrage profits 
for itself.
\end{abstract}

\tableofcontents
\section{Introduction}
With the advent of Layer 2 networks, subnets, supernets etc, 
we expect to see DEXes
that serve local economies without the need for bridging.
In these DEXes we expect to see only a small number of different tokens
with a handful of pools containing significant liquidity.
In such cases a DEX contract can check for arbitrage 
in a small set of predetermined potential opportunities.
This paper describes how this can be done efficiently.

\section{Setup and Notation}
We assume the DEX consists solely of UniswapV2-style pools.
Each pool $i$ is described by the tokens $s_i,t_i$ and 
the respective reserves $p_i, q_i$.
We let $R$ be the mapping from token pairs to reserves so
that for every pool $i$ we have $R(s_i, t_i) = p_i, q_i$
and $R(t_i, s_i) = q_i, p_i$.

The DEX administrator has identified a set of opportunities
that are simply lists/paths of tokens $[a,b,c, \ldots,a]$ such that
the first and last token are the same and for every adjacent
pair in the list $(a,b), (b,c), \ldots$ the exists a pool
in the DEX. When a user trades with the DEX the DEX executes 
the trade the same way that UniswapV2 does so. As part of the
same transaction however, the DEX subsequently checks if any
arbitrage opportunities are available, and if so executes 
the arb and gives the profits to stakers.

\section{Main Theorem}

For the case of all UniswapV2 pools where the trading fee is 
$(1-\gamma)$ (with $\gamma = 0.997$ in most deployments) 
we derive a closed form expression that tells us for any 
input amount $x_0$ the amount
of tokens $x_n$ we would get if we traded along 
the path $[a_0,a_1,a_2,\ldots, a_n]$ i.e.\ trade 
$x_0$ tokens of $a_0$ for $x_1$ tokens of $a_1$,
then $x_1$ tokens of $a_1$ for $x_2$ tokens of $a_2$ and so on
until we get $x_n$ tokens of $a_n$. The following theorem 
shows that $x_t$ has a form that can be updated easily.
In the next section we use this observation to compute 
the optimal amount $x_0$ to maximize profits on
paths of the form $[a_0, a_1, a_2,\ldots a_0]$

\begin{thm}
Given a DEX of UniswapV2 pools with trading fee $1-\gamma$ and 
a path of tokens $[a_0, a_1, \ldots, a_n]$, the output amount
$x_n$ satisfies
\begin{equation}
x_n = \frac{K_n x_0}{x_0 + M_n}
\end{equation}
where $K_n, M_n$ are defined 
via the recurrences
\begin{align}
K_{t} &= \frac{\gamma q_t K_{t-1}}{p_t + \gamma K_{t-1}} \\
M_{t} &= M_{t-1} - \frac{\gamma M_{t-1} K_{t-1}}{p_t + \gamma K_{t-1}}
\end{align}
where $p_t,q_t=R(a_{t-1},a_{t})$
are the reserves of 
tokens $a_{t-1}$ and $a_{t}$ in their pool.
and with starting values
$K_1, M_1 = q_1, p_1/\gamma$.
\end{thm}
\begin{proof}
The proof is by induction on the path length. For $n=1$
we have $p_1,q_1=R(a_0,a_1)$, $K_1= q_1$, $M_1 = p_1/\gamma$
and $x_1 = \frac{q_1 x_0}{x_0 + p_1/\gamma}$
We know check that the Uniswap invariant is maintained with the 
new reserves
\[
(p_1+\gamma x_0)(q_1 -x_1) = (p_1+\gamma x_0)\left(q_1 -\frac{\gamma q_1 x_0}{\gamma x_0 + p_1}\right) = (p_1+\gamma x_0)q_1 - \gamma q_1 x_0 = p_1q_1 
\]
This proves the base case. We now assume that the claim holds for paths 
of length $n-1$ and show that it holds for paths of length $n$.
By the induction hypothesis when we trade the path $[a_0,a_1,\ldots,a_{n-1},a_n]$ we know that we have
\[
x_{n-1} = \frac{K_{n-1} x_0}{x_0 + M_{n-1}}
\]
amount of $a_{n-1}$. Now let $p_n,q_n=R(a_{n-1},a_n)$ and consider
the Uniswap invariant again
\[
p_n q_n = (p_n + \gamma x_{n-1})(q_n - x_n).
\]
We see that $x_n$ must satisfy
\[
x_n = \frac{\gamma q_n x_{n-1}}{p_n + \gamma x_{n-1}}
\]
Plugging in the value of $x_{n-1}$ we have
\[
x_{n} =  \frac{\frac{\gamma q_n K_{n-1} x_0}{x_0 + M_{n-1}}}{p_n + \gamma \frac{K_{n-1} x_0}{x_0 + M_{n-1}}}
\]
\[
x_{n} =  \frac{\gamma q_n K_{n-1} x_0}{p_n(x_0 + M_{n-1}) + \gamma K_{n-1} x_0}
\]
\[
x_{n} =  \frac{\gamma q_n K_{n-1} x_0}{(p_n + \gamma K_{n-1} ) x_0 + p_n M_{n-1}}
\]
\[
x_{n} =  \frac{\frac{\gamma q_n K_{n-1}}{p_n + \gamma K_{n-1}} x_0}
{x_0 + \frac{p_n M_{n-1}}{p_n + \gamma K_{n-1}}}
\]
\[
x_{n} =  \frac{\frac{\gamma q_n K_{n-1}}{p_n + \gamma K_{n-1}} x_0}
{x_0 + M_{n-1}-\frac{\gamma M_{n-1} K_{n-1}}{p_n + \gamma K_{n-1}}}
\]
\[
x_{n} =  \frac{K_{n} x_0}{x_0 + M_{n}}
\]
where in the last step we used the recurrences for $K_{n}, M_{n}$. 
\end{proof}

\section{Optimal Arbitrage Amount}

Given a cycle $[a_0,a_1,\ldots,a_{n-1},a_0]$ we can solve for the 
optimal arbitrage amount by setting to zero the derivative of the profit
\[
y(x_0) = x_{n} - x_0 = \frac{K_{n} x_0}{x_0 + M_{n}} - x_0
\]
with respect to the input amount $x_0$. Doing so leads to the 
solution
\[
x_0^* = \sqrt{K_n M_n}-M_n
\]
For an arb to be possible we need $x_0^*>0$ or, equivalently, $K_n>M_n$. Furthermore the optimal profit is
\[
y(x_0^*) = K_n - M_n - 2 x_0^*
\]

\section{Algorithm}
Let $g(x,p,q, \gamma)$ be the amount that we receive when 
we swap $x$ amount of tokens with reserve $p$ for the other
token in the pool with reserve $q$ with the fee being $(1-\gamma)$, i.e:
\[
g(x, p, q, \gamma) = \frac{\gamma q x}{p+\gamma x}
\]
Below is a basic algorithm. Gas optimizations are up to the implementation.  
\begin{algorithmic}
\Require{List of opportunity paths $a$}
\For{$i \gets 0, \ldots, a.\textrm{length}-1$}
    \State $p, q \gets R(a[i][0],a[i][1])$
    \State $K, M \gets q, p/\gamma$
    \For{$j \gets 2, \ldots, a[i].\textrm{length}-1$}
    \State $p, q \gets R(a[i][j-1],a[i][j])$
    \State $M \gets M - g(K, p, M, \gamma)$
    \State $K \gets g(K, p, q, \gamma)$
    \EndFor
    \If{$K > M$} 
        \State $x \gets \sqrt{KM}-M$
        \State Swap $x$ tokens along path $a[i]$
        \State break
    \EndIf
\EndFor
\end{algorithmic}

\section{Additional Notes}
\subsection{Opportunities}
Suppose we have identified the cycle $[a_0,a_1,a_2,a_0]$ 
as an interesting opportunity. Should we also consider 
the cycles $[a_1,a_2,a_0,a_1]$ and $[a_2,a_1,a_0,a_2]$
which consist of the same pools? The answer is no: 
All these cycles allow one to
extract the same amount of value. 
The difference is only on the token being paid out.
While cyclical shifts need not be considered,
the reverse order cycle $[a_0,a_2,a_1,a_0]$ 
should always be considered.


%\vskip 0.2in
%\bibliography{sample}
%\bibliographystyle{unsrtnat}

\end{document}
